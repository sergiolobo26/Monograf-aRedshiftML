\chapter*{\centering \begin{normalsize}Abstract\end{normalsize}}
\begin{quotation}
	\noindent The DESI (Dark Energy Spectroscopic Instrument) project will conduct a five-year survey designed to cover 14.000 $deg^2$ by studying baryon acoustic oscillations (BAO) and redshift-space distortions (RSD). DESI needs simulations for its design, development, and operation, it uses simulations to evaluate the data pipelines and measurements of redshifts from the spectrometers. In particular, DESI uses a simulated survey of mock galaxies to compare their redshifts with the -also simulated- redshift measurements of DESI. These redshift measurements, however, present differences with respect to the true redshifts values of the mock galaxies. It is necessary to correct these measurements so that the instrument can work properly when tested in the real world. 
	
	The objective of this monograph is, therefore, to apply machine learning (ML) methods to the simulation data to recover the true redshift measurements of the Bright Galaxy Sample using observational variables as input. 
	
	First, we pre-process the data and select the variables as input for the ML models, these variables are the fluxes from the different observation bands, which are the $g, r, z, WISE 1,$ and $WISE 2$ bands, as well as the redshift output of the instrument's pipeline. The true redshift from the mock galaxies is used as output. We train and test support vector regression (SVR) and kernel ridge regression (KRR) models using cross-validation and grid search, later we use an ensemble model of different KRRs, this results in the best model with a coefficient of determination, $r^2 = 0.98$. This could be used to improve the accuracy of DESI when working in the five-year survey. Future work can be done in the fine-tuning of the model hyper-parameters, the trial of other model classes and different data preprocessing methods, as well as the application of this results to different classes of targets in the survey.      
\end{quotation}
\clearpage