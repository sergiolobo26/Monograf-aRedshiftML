\chapter{Introduction}

TODO Introduction and motivation of the project.

\section{Objectives}

\subsection{General Objective}
\begin{itemize}
	\item Determine the relation between the redshift simulated mesurements by DESI   and the intrinsic (simulated) redshift of BGS galaxies, to calibrate the instrument for real-world use using kernel method of machine learning
\end{itemize}

\subsection{Specific Objective}
\begin{itemize}
	\item Characterize the dataset using a simple statistical and graphical procedure to select the set of meaningful features as input to the machine learning algorithms. 
	\item Determine the set of computational parameters such as memory requirement, number of processors per node and size of dataset that performs the best on the cluster of the university restricted to constraints of resources, execution time and waiting time in queue.
	\item Train and adjust the hyper-parameter of the models by using grid-search and cross validation.
	\item Select the best model based on performance on unseen data and model simplicity.
\end{itemize}

\section{Context}
TODO Important concepts and definition from the DESI project and Machine learning
\subsection{DESI Project}
TODO Description of the Desi project and instrument, how the results are simulated. 
\subsection{Machine Learning}
TODO Supervised Learning, regression and classification, principal method of ML:
\subsubsection{Kernel Methods}
TODO Introduction, mathematics, and parameters of the model. 
\subsubsection{Neural Networks}
TODO Introduction, mathematics, and parameters of the model. 

conclusions about the complexity of the neural networks. 
