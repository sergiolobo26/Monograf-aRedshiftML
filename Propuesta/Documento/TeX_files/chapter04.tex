\chapter{Conclusions}
\label{Ch:conclusion}

It was found that for the BGS dataset that the variables FLUX\_G, FLUX\_R, FLUX\_Z, FLUX\_W1, FLUX\_W2 and Z (target Z), can be used effectively to approximate the instrument measurements to the ground true using kernel methods. Using a ensemble model consisting of three KRR (Kernel Ridge Regression) trained on subsamples of size 100.000, we were capable of aproximate the DESI instrument redshift measurements to its true value up to a $r^2 = 0.98$.  This model shows great potential to further enhancement, i.e. the fine tunning of the model hyper-parameters, different preprocessing methods, or even different classes of machine learning algorithms.

It is important to execute a training time evaluation as a function of the cluster resources and the parallelization algorithms, in order to know from the beginning the expected times of the algorithms and the correct selection of the sub-dataset size.  

Although the trained models didn´t work for other clases of galaxies, probably because the different distributions of fluxes, we infer that the same methodology can be used starting by training the models in the corresponding datasets. Moreover, a single model can be used for all the classes of targets by adding the categorical features of TYPE and SUBTYPE. We expected that by doing so, the results of this work can be replicated and applied to all the targets in the DESI measurements. 